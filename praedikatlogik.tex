\ifx\preampleIncluded\undefined
\def\startPraedikatlogik{}
\documentclass[12pt,a4paper,danish,twoside,reqno]{unf-compendium}

\usepackage[utf8]{inputenc}
\usepackage[danish]{babel}

\usepackage{amsmath, amsthm, amsfonts, amssymb}
\usepackage{bussproofs}
\usepackage{boxproof}
\usepackage{environ}
\usepackage{wasysym}
\usepackage{cleveref}
%\newtheorem{thm}{S{\ae}tning}
%\newtheorem{lem}{Lemma}
%\newtheorem{cor}{Korollar}

\theoremstyle{definition}
\newtheorem{dfn}{Definition}

\theoremstyle{remark}
\newtheorem{eks}{Eksempel}
\renewcommand{\phi}{\varphi}

\newcommand{\nats}{\ensuremath{\mathbb{N}}}
\newcommand{\reals}{\ensuremath{\mathbb{R}}}
\newcommand{\integers}{\ensuremath{\mathbb{Z}}}

\newcommand{\dx}{\,\mathrm{d} x}
\newcommand{\dz}{\,\mathrm{d} z}

\newcommand{\Uline}[1]{\underline{\underline{#1}}}
\newcommand{\vdashv}{\dashv\vdash}
\renewcommand{\imp}{\Rightarrow}
\newcommand{\bimp}{\Leftrightarrow}
\newcommand{\andL}{\wedge}
\newcommand{\orL}{\vee}
%\newcommand{\abs}[1]{\left|#1\right|}

\DeclareMathOperator{\Pfar}{far}
\DeclareMathOperator{\Pmor}{mor}
\DeclareMathOperator{\Pforaelder}{forælder}
\DeclareMathOperator{\Psoeskende}{søskende}
\DeclareMathOperator{\Phs}{hs}
\DeclareMathOperator{\Pfarmor}{farmor}
\DeclareMathOperator{\Pbedstemor}{bedstemor}
\DeclareMathOperator{\Pbedsteforaelder}{bedsteforælder}

\DeclareMathOperator{\Pvfu}{vfu}
\DeclareMathOperator{\pct}{\%}
\DeclareMathOperator{\BevisPar}{BevisPar}
\DeclareMathOperator{\Sub}{Sub}
\DeclareMathOperator{\Quine}{Quine}

\newif\ifsolution
\NewEnviron{solution}{\ifsolution{\par\noindent\textbf{Løsning.} }\expandafter\BODY\hfill$\square$\par\fi}
\theoremstyle{definition}\newtheorem{exercise}{Opgave}

\def\preampleIncluded{}

\begin{document}
\fi

\section{Prædikatlogik}
I det foregående har vi set hvordan vi formelt kan beskrive simple udsagn og sammensætninger heraf. Hvis vi vil begynde at snakke om egenskaber ved samlinger af ting kommer vi imidlertid til kort her. Et udsagn som "Alle matcamp deltagerne er gode til logik" kan vi f.eks. godt se som et udsagn i udsagnslogik. "Der findes en matcamp deltager der er god til logik" er et andet udsagn, som i udsagnslogik ikke umiddelbart har nogen relation til det første. Men vores intuition fortæller os at hvis det første er sandt, så må det andet også være det. Vores mål her er at definere et "rigere" sprog der gør det muligt for os at analysere sammenhængende mellem udsavn om elementer i samlinger. Vi kalder dette sprog for "prædikatlogik"

Basis for prædikatlogikken er hvad vi kalder for kvantiseret udtryk:
\[
	\forall x \, P(x)
\]

Det læses som: "For alle $x$ gælder $P(x)$". $\forall$ symbolet kaldet for en "alkvantor". $x$ her er en variabel, og $P(x)$ er hvad vi kalder for et prædikat -- et prædikat er et udsagn der afhænger af en eller flere variable. Præcist hvad variable og prædikater kan være beskæftiger prædikatlogikken sig ikke direkte med, det er noget man definerer ved at tilføje flere aksiomer til ens logiske system.

Så hvad mener vi så når vi siger "alle $x$"? Det kommer an på hvad vi ellers har defineret -- eneste det siger er at for enhver variabel vi på en eller anden måde kan komme frem til skal udagnet $P(x)$ være sandt. I matematik har i måske set et lignende udtryk i form af
\[
	\forall n \in \nats : 2n \geq n
\]

Hvis man ikke er bekendt med et sådan udtryk, så er det der står simpelthen den trivielle sætning at hvis man fordobler et naturligt tal så får man et tal der er større end eller lig med tallet. Denne måde at skrive kvantiseret udtryk på kan ses som en "syntaktisk genvej" til vores måde at skrive kvantiseret udtryk på i prædikatlogik. F.eks. kan ovestående i prædikatlogik skrives som: 
\[
	\forall n \, n \in \nats \imp 2n \geq n
\]

Dvs. "For alle $n$ gælder at når $n$ er et naturligt tal så er $2n$ større eller lig med $n$".
Hvis vi ved at noget gælder for alle elementer i en samling må vi også kunne sige at det gælder for enkelte af dem.
F.eks. hvis vi har sætningen "alle stjerner er mere end 1000 km væk"{ }så skulle vi nok også kunne konkludere at solen er mere end 1000 km væk.
Vi introducerer derfor vores første aksiom "for alle eliminering":
\begin{prooftree}
	\AxiomC{$\forall x \, \phi$}
	\RightLabel{$\forall x$ e}
	\UnaryInfC{$\phi[t/x]$}
\end{prooftree}
Her betyder $\phi[t/x]$ at vi tager $\phi$ og erstatter med $t$ alle forekomster af $x$ der er frie for $t$.
Og hvad betyder det så? Lad os tage udsagnet $\forall x \exists y \, x < y$. Det er naturligvis sandt for f.eks. reelle tal.
Hvis vi nu bruger $\forall x$ e hvor vi erstatter $x$ med $y$ så får vi $\exists y \, y < y$. Nu kan et tal ikke være mindre end sig selv, så det er noget vås vi får ud af det.
Det er derfor vigtigt at vi ikke substituerer med en variabel der er bundet til en kvantor.
I det her tilfælde ville vi sige at $x$ ikke var fri for $y$ -- altså $x$ kan ikke erstattes med $y$ fordi $x$ fremgår i et prædikat hvor $y$ er bundet af en kvantor.

Det vi betegner som "variable" her er en delmængde at hvad vi kalder "termer".
Et term er et symbol som virker som en slags udskiftelig del af et udtryk. Hvis vi skal være meget pendantiske så giver vi en samling af symboler som vi siger er gyldige termer. I praksis er det eneste vigtige at de ikke syntaktisk kan "kollidere" med symboler der ellers har betydning. F.eks. ville
"$\phi \land \psi$" være et dårligt valg af navn til et term.

Rent formelt set er der ikke nogen distinktion mellem termer og variable i form af tegnene selv, den findes kun i hvordan vi bruger dem.
F.eks. er et udtryk som $\forall 0 \, 0=0$ ækvivalent med $\forall x \, x=x$, men det siger ikke noget om tallet 0 selv hvis vi har andre sætninger i vores system der definerer 0, fordi det eneste man kan gøre med 0'et her er at erstatte det med en anden term ved brug af $\forall 0$ e.

Når vi læser reglerne for kvantorene skal vi forstå variablene der står der som placeholders for ethvert term -- altså kan de $x$ og $x_0$ der står ved reglerne for kvantorene erstattes med ethvert andet term -- specielt må de gerne erstattes af det samme term.
F.eks. hvis vi har $\vdash 0+1=1$ kan vi ikke erstatte termen 0 med termen 1, men hvis der står $\forall 0 \, 0=0$ må vi gerne konkludere $1=1$. 
Hvis vi skulle være meget pendantiske skulle vi faktisk angive ved reglerne hvilke termer der fungerer som variable.

Udover alkvantoren har vi også "eksistenskvantoren": $\exists$. Et kvantiseret udtryk med denne har formen
\[
	\exists x \, P(x),
\]
hvilket vil sige "Der eksisterer et $x$ hvor $P(x)$ gælder". F.eks. kunne vi have udtrykket "der findes en bilejer": $\exists x \, x\text{ ejer en bil}$. Hvordan finder vi så ud af at dette er sandt? En måde kunne jo være at bevise det ved et eksempel. Lad os f.eks. sige at vi finder en bilejer der hedder Søren. Så skal vi fra sætningen "Søren ejer en bil" kunne konkludere $\exists x \, x\text{ ejer en bil}$. Så vi kunne jo passende introducere et axiom der tillader os dette:
\begin{prooftree}
	\AxiomC{$\phi[t/x]$}
	\RightLabel{$\exists x $ i}
	\UnaryInfC{$\exists x \, \phi$}
\end{prooftree}

Hvad så med at introducere for alle? Hvis vi skal bevise at noget gælder for alle elementer i en samling så starter vi med at antage at vi er givet et af elementerne hvorefter vi viser sætningen for dette. Hvis vi kan vise at sætningen gælder lige meget hvilket element vi er givet så vil det være oplagt at konstatere at det gælder for alle elementer. Derfor introducere vi "for alle introduktion":
\begin{prooftree}
	\AxiomC{$$\boxed{
		\begin{matrix}
			\multicolumn{1}{l}{x_0} \\
			\vdots \\
			\phi[x_0/x]
		\end{matrix}
	}$$}
	\RightLabel{$\forall x $ i}
	\UnaryInfC{$\forall x \, \phi$}
\end{prooftree}
Det venstrestillede $x_0$ betyder at vi har en variabel som vi bruger inde i boksen. Denne variabel må ikke forekomme noget andet sted i beviset (i hvert fald noget sted der er refereret til direkte eller indirekte inden i boksen). Det tilsvarer at man i matematik vil bevise at noget gælder for alle af noget ved at man starter med at lade et $x$ være givet.

Vi har nu for-alle-introduktion og eliminering og der-eksisterer-introduktion. Det vi nu mangler er en regel for der-eksisterer-eliminering ($\exists x$ e). Hvis vi ved at der eksisterer et element der opfylder en eller anden egenskab og gerne vil vise at en sætning gælder ud fra det, så skal vi vise at sætningen gælder uanset hvilken værdi det er der opfylder egenskaben. Så vi introducere "der eksisterer introduktion"{} ved:
\begin{prooftree}
	\AxiomC{$$\boxed{
		\begin{matrix}
			\multicolumn{1}{l}{x_0 \quad \phi[x_0/x]} \\
			\vdots \\
			\chi
		\end{matrix}
	}$$}
	\RightLabel{$\exists x $ e}
	\UnaryInfC{$\chi$}
\end{prooftree}
Altså lader vi der være en arbitrær variabel vi siger prædikatet er opfyldt for, og så viser vi at der er en sætning $\chi$ der er opfyldt \textit{som ikke selv indeholder variablen}.

Hvis der for alle elementer gælder en ting må det være sandt at det ikke gælder at der findes et element for hvilket tingen ikke gælder. Omvendt hvis der eksisterer et element hvor nået gælder for så må det være sandt at det ikke gælder at for alle elementer er den ting ikke opfyldt. Udtrykt i prædikatlogik vil det sige at sætningerne
\[
	\forall x \, \phi \bimp \neg \exists x \neg \phi
\]
og
\[
	\exists x \, \phi \bimp \neg \forall x \neg \phi
\]
er sande. Vi viser den første af dem her:
\begin{proofbox}
   \lbl{imp_box1}
   \[
      \: \forall x \, \phi \= \text{antagelse (for $\imp$i)} \\
      \lbl{neg_box}
      \[
      	\: \exists x \neg \phi \= \text{antagelse (for $\neg$i)} \\
      	\lbl{exists_box}
      	\[
 	   		\lbl{1_1}
 	   		x_0 \: \neg\phi[x_0/x] \= \text{antagelse (for $\exists x$ e)} \\
      		\lbl{1_2}
      		\: \phi[x_0/x] 	\= \forall x \text{ e} \\
      		\label{_forall_to_not_box3a_end}
      		\: \bot \= \neg \text{ e } \ref{1_2},\ref{1_1}
      	\]
      	\label{_forall_to_not_box2a_end}
      	\: \bot \= \exists x \text{ e } \ref{exists_box}-\ref{_forall_to_not_box3a_end}
      \]
      \label{_forall_to_not_box1a_end}
      \: \neg \exists x \neg \phi \= \neg \text{ i } \ref{neg_box}-\ref{_forall_to_not_box2a_end}
   \]
  \lbl{imp_1}
  \: \forall x \, \phi \imp \neg \exists x \neg \phi \= \imp \text{i }
    \ref{imp_box1}-\ref{_forall_to_not_box1a_end} \\
  \lbl{imp_box2}
  \[
    \lbl{2}
  	\: \neg \exists x \neg \phi \= \text{antagelse (for $\imp$ i)} \\
  	\lbl{forall_box}
  	\[
  		x_0 \= \text{lad $x_0$ være givet (for $\forall x$ i)} \\
  		\lbl{pbc_box}
  		\[
  		    \lbl{2_1_1}
  			\: \neg \phi[x_0/x] \= \text{antagelse (for BVM)} \\
  			\lbl{2_1_2}
  			\: \exists x \neg \phi \= \exists \text{ i } \ref{2_1_1} \\
  			\label{_forall_to_not_box3b_end}
  			\: \bot \= \neg \text{e } \ref{2_1_2}, \ref{2}
  		\]
  		\label{_forall_to_not_box2b_end}
  		\: \phi[x_0/x] \= \text{BVM } \ref{pbc_box}-\ref{_forall_to_not_box3b_end}
  	\]
  	\label{_forall_to_not_box1b_end}
  	\: \forall x \, \phi \= \forall x \text{ i } \ref{forall_box}-\ref{_forall_to_not_box2b_end}
  \]
  \lbl{imp_2}
  \: \neg \exists x \neg \phi \imp \forall x \, \phi \= \imp \text{i } \ref{imp_box2}-\ref{_forall_to_not_box1b_end} \\
  \: \forall x \, \phi \bimp \neg \exists x \neg \phi \= \bimp \text{i } \ref{imp_1},\ref{imp_2}
\end{proofbox}

Vi har nu introduceret nogen regler der gør det muligt at arbejde med prædikater der er opfyldt for elementer i hele eller dele af vores univers. Men vi mangler en måde at snakke om sammenhænge imellem elementerne. Vi introducere derfor "$=$". Vi starter med introduktionsreglen:

\begin{prooftree}
	\AxiomC{}
	\RightLabel{$=$i}
	\UnaryInfC{$t=t$}
\end{prooftree}

Altså er ting lig med sig selv, hvilket ikke skulle være nogen overaskelse. $t$ her er igen en placeholder for en arbitrær term. Vi kan ikke rigtigt bruge den regel til så meget i sig selv, derfor har vi også en elimineringsregel:

\begin{prooftree}
	\AxiomC{$t_1 = t_2$}
	\AxiomC{$\phi[t_1/x]$}
	\RightLabel{$=$e}
	\BinaryInfC{$\phi[t_2/x]$}
\end{prooftree}

Her er også $t_1,t_2$ og $x$ placeholders for andre termer. Det kan være svært at se den egentlige styrke af reglen, tricket ved den er ofte at man skal finde den rigtige $\phi$ at bruge -- bemærk nemmelig at det er to forskellige termer der er erstattet med i præmissen og i konklusionen. Vi kan bruge den til at konkludere nogen af de egenskaber vi forventer gælder for lig med. Vi vil gerne have at hvis $t_1=t_2$ og $t_2=t_3$ så er $t_1=t_2$. Altså at der gælder $t_1=t_2,t_2=t_3\vdash t_1=t_3$.Vi kalder dette transitivitet.
\begin{proofbox}
	\lbl{1}
	\: t_1 = t_2 \= \text{præmis} \\
	\lbl{2}
	\: t_2 = t_3 \= \text{præmis} \\
	\: t_1 = t_3 \= \text{$=$e \ref{2},\ref{1}: $\phi\!: t_1=t, [t_2/t]\to[t_3/t]$}
\end{proofbox}

Derudover forventer vi også at lig med er symmetrisk. Dvs. at hvis $a=b$ så er $b=a$, så vi viser $t_1=t_2 \vdash t_2=t_1$:
\begin{proofbox}
	\lbl{1}
	\: t_1 = t_2 \\
	\lbl{2}
	\: t_1 = t_1 \= \text{$=$i} \\
	\: t_2 = t_1 \= \text{$=$e \ref{1},\ref{2}: $\phi\!: t=t_1, [t_1/t]\to[t_2/t]$}
\end{proofbox}

%\begin{prooftree}
%	\AxiomC{$$\boxed{
%		\begin{matrix}
%			\phi \\
%			\vdots \\
%			\psi
%		\end{matrix}
%	}$$}
%	\UnaryInfC{$\exists x \, \phi$}
%\end{prooftree}

\ifdefined\startPraedikatlogik\end{document}\fi
