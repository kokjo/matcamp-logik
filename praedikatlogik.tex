I det foregående har vi set hvordan vi formelt kan beskrive simple udsagn og sammensætninger heraf. Hvis vi vil begynde at snakke om egenskaber ved samlinger af ting kommer vi imidlertid til kort her. Et udsagn som "Alle matcamp deltagerne er gode til logik" kan vi f.eks. godt se som et udsagn i udsagnslogik. "Der findes en matcamp deltager der er god til logik" er et andet udsagn, som i udsagnslogik ikke umiddelbart har nogen relation til det første. Men vores intuition fortæller os at hvis det første er sandt, så må det andet også være det. Vores mål her er at definere et "rigere" sprog der gør det muligt for os at analysere sammenhængende mellem udsavn om elementer i samlinger. Vi kalder dette sprog for "prædikatlogik"

Basis for prædikatlogikken er kvantiseret udtryk:
\[
	\forall x : P(x)
\]

Det læses som: "For alle $x$ gælder $P(x)$". $\forall$ symbolet kaldet for en "al-kvantor". $x$ her er en variabel, og $P(x)$ er hvad vi kalder for et prædikat - et prædikat er et udsagn der afhænger af en eller flere variable. Præcist hvad variable og prædikater kan være beskæftiger prædikatlogikken sig ikke direkte med, det er noget man definerer ved at tilføje flere aksiomer til ens logiske system.

Så hvad mener vi så når vi siger "alle $x$"? Det kommer an på hvad vi ellers har defineret - eneste det siger er at for enhver variabel vi på en eller anden måde kan komme frem til skal udagnet $P(x)$ være sandt. I matematik har i måske set et lignende udtryk i form af
\[
	\forall n \in \nats : 2n \leq n
\]

Hvis man ikke er bekendt med et sådan udtryk, så er det der står simpelthen den trivielle sætning at hvis man fordobler et naturligt tal så får man et tal der er større end eller lig med tallet. Det har 
