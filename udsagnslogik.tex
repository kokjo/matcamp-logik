Udsagnslogik bruges til at beskrive logiske strukture uafhænigt af virkeligheden.
Eksempler fra virkeligheden:
\begin{itemize}
    \item ``Det regner''
    \item ``Det er ikke sandt at det ikke regner''
    \item ``Hvis jeg regner bliver jeg våd''
    \item ``Hvis det regner bliver jeg våd. Det regner, derfor bliver jeg våd''
\end{itemize}

For at beskrive sådanne udsagn har vi brug for en del notation notation. Et udsagn vil fra nu af blive beskrevet ved brug af af små bøgstaver: $p, q, \cdots$,
vi sammensætter udsagns ved hjælp at konnektiver, til disse bruger vi notationen:

\begin{description}
    \item[$\land$] betyder 'og', det vil sige at $a \land b$ er sand hvis $a$ og $b$ er sande.
    \item[$\lor$] betyder 'eller', det vil sige at $a \lor b$ er sand hvis $a$, $b$ eller begge er sande.
    \item[$\lnot$] betyder 'ikke'(læses non), det vil sige at $\lnot a$ er sand, hvis $a$ er falsk.
    \item[$\imp$] betyder 'medføre', det vil sige at $a \imp b$ er equvivalent med $\lnot a \lor b$.
    \item[$\bimp$] betyder 'ækvivalent', det vil sige at $a \bimp b$ er sand, hvis $a$ og $b$ har samme sandheds værdi.
\end{description}

Vi kan nu fjerne alt unødig information fra de eksempler vi havde før:
\begin{itemize}
    \item ``$a$''
    \item ``$\lnot \lnot a$''
    \item ``$a \imp b$''
\end{itemize}

Den sidste sætning kan vi dog endnu ikke reducere til symboler, derfor bliver vi nød til at indføre mere notation:

\begin{prooftree}
    \AxiomC{Ting vi antager}
    \UnaryInfC{Ting vi har lov til at konkludere}
\end{prooftree}

For at kunne analysere sådanne udtryk, er det nødvendigt at indføre et formelt sprog hvor vi kan beskrive sådanne strukture.
Logik kan beskrives ved simple substitutioner,  vi vil beskrive disse substitioner med følgene notation:

Så vi kan indføre
