\section{Udsagnslogik}
Udsagnslogik bruges til at beskrive logiske strukture uafhænigt af virkeligheden.
Eksempler fra virkeligheden:
\begin{itemize}
    \item ``Det regner''
    \item ``Det er ikke sandt at det ikke regner''
    \item ``Hvis det regner bliver jeg våd''
    \item ``Hvis det regner bliver jeg våd. Det regner, derfor bliver jeg våd''
\end{itemize}

For at beskrive sådanne udsagn har vi brug for en del notation notation. Et udsagn vil fra nu af blive beskrevet ved brug af af små bøgstaver: $p, q, \cdots$,
eller små græske bogstaver: $\phi$, $\psi$, \ldots
vi sammensætter udsagns ved hjælp at konnektiver, til disse bruger vi notationen:

\begin{description}
    \item[$\land$] betyder 'og', det vil sige at $a \land b$ er sand hvis $a$ og $b$ er sande.
    \item[$\lor$] betyder 'eller', det vil sige at $a \lor b$ er sand hvis $a$, $b$ eller begge er sande.
    \item[$\lnot$] betyder 'ikke'(læses non), det vil sige at $\lnot a$ er sand, hvis $a$ er falsk.
    \item[$\imp$] betyder 'medføre', det vil sige at $a \imp b$ er equvivalent med $\lnot a \lor b$. Et udtryk $a \imp b$ kaldes for en 'implikation'.
    \item[$\bimp$] betyder 'ækvivalent med', det vil sige at $a \bimp b$ er sand, hvis $a$ og $b$ har samme sandheds værdi. Et udtryk $a \bimp b$ kaldes for en 'biimplikation'.
\end{description}



Vi kan nu fjerne alt unødig information fra de eksempler vi havde før:
\begin{itemize}
    \item ``$a$''
    \item ``$\lnot \lnot a$''
    \item ``$a \imp b$''
\end{itemize}

Den sidste sætning kan vi dog endnu ikke reducere til symboler, derfor bliver vi nød til at indføre mere notation:

\begin{prooftree}
    \AxiomC{Ting vi antager}
    \UnaryInfC{Ting vi har lov til at konkludere}
\end{prooftree}

Vi skal også bruge notationen $p \vdash q$ for at sige at p beviser q(der eksistere et bevis som antager p og konkludere q, uden rent faktisk at give et bevis for det).

Nu kan vi så antage at $a \imp b$ og $a$ og derefter konkludere $b$:
\begin{prooftree}
    \AxiomC{$a \imp b$}
    \AxiomC{$a$}
    \BinaryInfC{$b$}
\end{prooftree}

Vi kan med denne notation kan vi indføre regler for hvad vi kan konkludere udfra sammesatte udsagn.
F.eks. kan reglerne for brugen af 'og' sammensætnigen defineres som 3 regler:

\begin{prooftree}
    \AxiomC{$\phi$}
    \AxiomC{$\psi$}
    \RightLabel{($\land$i)}
    \BinaryInfC{$\phi \land \psi$}
\end{prooftree}
Reglen siger at hvis vi ved $\phi$ og $\psi$ er sande, ved vi også at $\phi \land \psi$ er sand(vi har bare lavet 'og' om til $\land$). Reglen kaldes introduktion af 'og'

\begin{prooftree}
    \AxiomC{$\phi \land \psi$}
    \RightLabel{($\land e_1$)}
    \UnaryInfC{$\phi$}
\end{prooftree}

\begin{prooftree}
    \AxiomC{$\phi \land \psi$}
    \RightLabel{($\land e_2$)}
    \UnaryInfC{$\psi$}
\end{prooftree}
Hvis vi ved at $\phi \land \psi$ er sand, ved vi specielt også at $\phi$ og $\psi$ er sande hver for sig. Disse 2 regler kaldes elimination af 'og'.

'eller'-reglerne er lidt sværer, for at introducere et $\lor$ er det kun nødvendigt at vide om en af 2 udsagn er sande:
\begin{prooftree}
    \AxiomC{$\phi$}
    \RightLabel{($\lor i_1$)}
    \UnaryInfC{$\phi \lor \psi$}
\end{prooftree}
\begin{prooftree}
    \AxiomC{$\psi$}
    \RightLabel{($\lor i_2$)}
    \UnaryInfC{$\phi \lor \psi$}
\end{prooftree}

Elimination af 'eller' er den del svære, hvis vi har lyst til at konkludere $\alpha$ udfra $\phi \lor \psi$, ved vi ikke om det er $\phi$, $\psi$ eller begge som er sande.
Vi bliver derfor nød til at skabe 2 seperate beviser hvor vi antager $\phi$ og $\psi$ seperat og konkludere $\alpha$ i begge tilfælde:

\begin{prooftree}
    \AxiomC{$\phi \lor \psi$}
    \AxiomC{$$\boxed{
        \begin{matrix}
            \phi \\
            \vdots \\
            \alpha
        \end{matrix}}$$}
    \AxiomC{$$\boxed{
        \begin{matrix}
            \psi \\
            \vdots \\
            \alpha
        \end{matrix}}$$}
    \RightLabel{($\lor e$)}
    \TrinaryInfC{$\alpha$}
\end{prooftree}

Reglerne for negation:
\begin{prooftree}
    \AxiomC{$$\boxed{
        \begin{matrix}
            \phi \\
            \vdots \\
            \bot
        \end{matrix}}$$}
    \RightLabel{($\lnot i$)}
    \UnaryInfC{$\lnot \phi$}
\end{prooftree}
Hvis du antager noget og kommer frem til at du kan konkludere falsk, må det du antog være falsk. Det omvendte er derfor sandt.

\begin{prooftree}
    \AxiomC{$\lnot \lnot \phi$}
    \RightLabel{($\lnot \lnot e$)}
    \UnaryInfC{$\phi$}
\end{prooftree}
``minus minus giver plus''

\begin{prooftree}
    \AxiomC{$$\boxed{
        \begin{matrix}
            \phi \\
            \vdots \\
            \psi
        \end{matrix}}$$}
    \RightLabel{($\imp$ i)}
    \UnaryInfC{$\phi \imp \psi$}
\end{prooftree}
Hvis du kan bevise $\psi$ udfra $\psi$ så $\phi \imp \psi$.

\begin{prooftree}
    \AxiomC{$\phi \imp \psi$}
    \AxiomC{$\phi$}
    \RightLabel{($\imp$ e)}
    \BinaryInfC{$\psi$}
\end{prooftree}
Hvis $\phi$ medføre $\psi$, og $\phi$ er sand, må $\psi$ også være sand.

Ækvivalent med ($\bimp$) er bare en forkortelse for at der er en implikation begge veje mellem udtrykkene. Det kan vi udtrykke ved reglerne
\par\noindent \begin{tabular}{@{}l@{}l@{}}
	\begin{minipage}{0.5\textwidth}
		\begin{prooftree}
		    \AxiomC{$\phi \imp \psi$}
		    \AxiomC{$\psi \imp \phi$}
		    \RightLabel{($\bimp$ i)}
		    \BinaryInfC{$\phi \bimp \psi$}
		\end{prooftree}
	\end{minipage}
	&
	\begin{minipage}{0.5\textwidth}
		\begin{prooftree}
		    \AxiomC{$\phi \bimp \psi$}
		    \RightLabel{($\bimp\text{ e}_\land$)}
		    \UnaryInfC{$\left(\phi \imp \psi\right) \land \left(\psi \imp \phi\right)$}
		\end{prooftree}
	\end{minipage}
\end{tabular}

\subsection{Beviser}
Da vi nu har alle reglerne på plads, kan vi bruge dem til at vise at andre sætninger er sande.
Hvis vi tager sætningen $p \imp q \vdash p \land r \imp q \land r$, ser vi at det ikke er en af vores axiomer,
Men vi kan udfra vores axiomer konkludere om sætningen er sand, ved at udføre et bevis for den:
\begin{prooftree}
    \AxiomC{$p \imp q$}
    \AxiomC{$p \land r$}
    \RightLabel{($\land e_1$)}
    \UnaryInfC{$p$}
    \RightLabel{($\imp$ e)}
    \BinaryInfC{$q$}
    \AxiomC{$p \land r$}
    \RightLabel{($\land e_2$)}
    \UnaryInfC{$r$}
    \RightLabel{($\land i$)}
    \BinaryInfC{$q \land r$} 
\end{prooftree}

Dette ser en smule kludret ud, derfor er det bedre at bruge en form for linært bevis som ser således ud:

%\begin{tabular}{l c r}
%    1 & $p \land r$ & premisse \\
%    2 & $p$         & $\land e_1 \, 1$ \\
%    3 & $r$         & $\land e_2 \, 1$ \\
%    4 & $p \imp q$  & antagelse \\
%    5 & $q$         & $\imp e \, 2, 4$ \\
%    6 & $q \land r$ & $\land i \, 3, 5$
%\end{tabular}

\begin{proofbox}
    \lbl{eks1}
    \: p \land r        \= \text{premisse} \\
    \lbl{eks1_p}
    \: p                \= \land e_1, 1 \\
    \lbl{eks1_r}
    \: r                \= \land e_2, 1 \\
    \lbl{eks1_p_imp_q}
    \: p \imp q         \= \text{antagelse} \\
    \lbl{eks1_q}
    \: q                \= \imp e \ref{eks1_p_imp_q}, \ref{eks1_p}
    \: q \land r        \= \land i \ref{eks1_q}, \ref{eks1_r}
\end{proofbox}

\subsection{Eksempler på beviser}
\subsubsection{$p \imp q, q \imp r \vdash p \imp r$}
\begin{proofbox}
    \lbl{eks2_imp_box}
    \[
        \lbl{eks2_p}
        p \: p      \= \text{premisse for $\imp$ i} \\
        \lbl{eks2_piq}
        \: p \imp q \= \text{antagelse} \\
        \lbl{eks2_q}
        \: q        \= \imp e \ref{eks2_p}, \ref{eks2_piq} \\
        \lbl{eks2_qir}
        \: q \imp r \= \text{antagelse} \\
        \label{_eks2_imp_box_end}
        \: r        \= \imp e \ref{eks2_q} \\
    \]
    \: p \imp r     \= \imp i \ref{eks2_imp_box}-\ref{_eks2_imp_box_end}
\end{proofbox}


\subsubsection{$p \vdash q \imp p$}
Beviset her er lidt sjovt da $p$ klart er sandt, men selvom vi intet viden har omkring $q$,
kan $q$ stadig blive inkluderet i et udsagn.
\begin{proofbox}
    \lbl{eks3_qip}
    \[
        \lbl{eks2_q}
        q \: q      \= \text{premisse for $\imp$ i} \\
        \label{_eks3_qip_end}
        \: p        \= \text{Antagelse} \\
    \]
    \: q \imp p     \= \imp i \ref{eks3_qip}-\ref{_eks3_qip_end}
\end{proofbox}


%\begin{tabular}[l c r]
%    \begin{prooftree}
%        \AxiomC{$\phi$}
%        \RightLable{($\lor i$)}
%        \UnaryInfC{$\phi \lor \psi$}
%    \end{prooftree} &
%\end{tabular}

