\ifx\preampleIncluded\undefined
\def\startUdsagnslogik{}
\documentclass[12pt,a4paper,danish,twoside,reqno]{unf-compendium}

\usepackage[utf8]{inputenc}
\usepackage[danish]{babel}

\usepackage{amsmath, amsthm, amsfonts, amssymb}
\usepackage{bussproofs}
\usepackage{boxproof}
\usepackage{environ}
\usepackage{wasysym}
\usepackage{cleveref}
%\newtheorem{thm}{S{\ae}tning}
%\newtheorem{lem}{Lemma}
%\newtheorem{cor}{Korollar}

\theoremstyle{definition}
\newtheorem{dfn}{Definition}

\theoremstyle{remark}
\newtheorem{eks}{Eksempel}
\renewcommand{\phi}{\varphi}

\newcommand{\nats}{\ensuremath{\mathbb{N}}}
\newcommand{\reals}{\ensuremath{\mathbb{R}}}
\newcommand{\integers}{\ensuremath{\mathbb{Z}}}

\newcommand{\dx}{\,\mathrm{d} x}
\newcommand{\dz}{\,\mathrm{d} z}

\newcommand{\Uline}[1]{\underline{\underline{#1}}}
\newcommand{\vdashv}{\dashv\vdash}
\renewcommand{\imp}{\Rightarrow}
\newcommand{\bimp}{\Leftrightarrow}
\newcommand{\andL}{\wedge}
\newcommand{\orL}{\vee}
%\newcommand{\abs}[1]{\left|#1\right|}

\DeclareMathOperator{\Pfar}{far}
\DeclareMathOperator{\Pmor}{mor}
\DeclareMathOperator{\Pforaelder}{forælder}
\DeclareMathOperator{\Psoeskende}{søskende}
\DeclareMathOperator{\Phs}{hs}
\DeclareMathOperator{\Pfarmor}{farmor}
\DeclareMathOperator{\Pbedstemor}{bedstemor}
\DeclareMathOperator{\Pbedsteforaelder}{bedsteforælder}

\DeclareMathOperator{\Pvfu}{vfu}
\DeclareMathOperator{\pct}{\%}
\DeclareMathOperator{\BevisPar}{BevisPar}
\DeclareMathOperator{\Sub}{Sub}
\DeclareMathOperator{\Quine}{Quine}

\newif\ifsolution
\NewEnviron{solution}{\ifsolution{\par\noindent\textbf{Løsning.} }\expandafter\BODY\hfill$\square$\par\fi}
\theoremstyle{definition}\newtheorem{exercise}{Opgave}

\def\preampleIncluded{}

\begin{document}
\fi

\section{Udsagnslogik}
Udsagnslogik bruges til at beskrive logiske strukture uafhænigt af virkeligheden.
Eksempler fra virkeligheden:
\begin{itemize}
    \item ``Det regner''
    \item ``Det er ikke sandt at det ikke regner''
    \item ``Hvis det regner bliver jeg våd''
    \item ``Hvis det regner bliver jeg våd. Det regner, derfor bliver jeg våd''
\end{itemize}

For at beskrive sådanne udsagn har vi brug for en del notation. Et udsagn vil fra nu af blive beskrevet ved brug af af små bøgstaver: $p, q, \cdots$,
eller små græske bogstaver: $\phi$, $\psi$, \ldots
vi sammensætter udsagn ved hjælp af symboler vi vil kalde konnektiver(symboler som sætter udsagn sammen), til disse bruger vi notationen:

\begin{description}
    \item[$\land$] betyder 'og', det vil sige at $a \land b$ er sand hvis $a$ og $b$ er sande.
    \item[$\lor$] betyder 'eller', det vil sige at $a \lor b$ er sand hvis $a$, $b$ eller begge er sande.
    \item[$\lnot$] betyder 'ikke'(læses non), det vil sige at $\lnot a$ er sand, hvis $a$ er falsk.
    \item[$\imp$] betyder 'medføre'. Et udtryk $a \imp b$ kaldes for en 'implikation'.
    \item[$\bimp$] betyder 'ækvivalent med', det vil sige at $a \bimp b$ er sand, hvis $a$ og $b$ har samme sandheds værdi. Et udtryk $a \bimp b$ kaldes for en 'biimplikation'.
\end{description}

Vi kan nu fjerne alt unødig information fra de eksempler vi havde før, og kun se på den logiske struktur:
\begin{itemize}
    \item ``$a$''
    \item ``$\lnot \lnot a$''
    \item ``$a \imp b$''
\end{itemize}

Den sidste sætning kan vi dog endnu ikke reducere til symboler, grunden til dette er at vi mangler en måde at konkludere derfor bliver vi nød til at indføre mere notation:

\begin{prooftree}
    \AxiomC{Ting vi antager}
    \RightLabel{(regelnavn)}
    \UnaryInfC{Ting vi har lov til at konkludere}
\end{prooftree}

Vi skal også bruge notationen $p \vdash q$ for at sige at p beviser q(der eksisterer et bevis som antager p og konkludere q, uden rent faktisk at give et bevis for det).

Nu kan vi så antage at $a \imp b$ og $a$ og derefter konkludere $b$:
\begin{prooftree}
    \AxiomC{$a \imp b$}
    \AxiomC{$a$}
    \BinaryInfC{$b$}
\end{prooftree}

Med denne notation kan vi indføre regler for hvad vi kan konkludere udfra sammesatte udsagn.
F.eks. kan reglerne for brugen af 'og' sammensætnigen defineres som 3 regler:

\begin{prooftree}
    \AxiomC{$\phi$}
    \AxiomC{$\psi$}
    \RightLabel{($\land$i)}
    \BinaryInfC{$\phi \land \psi$}
\end{prooftree}
Reglen siger at hvis vi ved $\phi$ og $\psi$ er sande, ved vi også at $\phi \land \psi$ er sand(vi har bare lavet 'og' om til $\land$). Reglen kaldes introduktion af 'og'

\begin{prooftree}
    \AxiomC{$\phi \land \psi$}
    \RightLabel{($\land e_1$)}
    \UnaryInfC{$\phi$}
\end{prooftree}

\begin{prooftree}
    \AxiomC{$\phi \land \psi$}
    \RightLabel{($\land e_2$)}
    \UnaryInfC{$\psi$}
\end{prooftree}
Hvis vi ved at $\phi \land \psi$ er sand, ved vi specielt også at $\phi$ og $\psi$ er sande hver for sig. Disse 2 regler kaldes elimination af 'og'.

'eller'-reglerne er lidt sværer, for at introducere et $\lor$ er det kun nødvendigt at vide om en af 2 udsagn er sande:
\begin{prooftree}
    \AxiomC{$\phi$}
    \RightLabel{($\lor i_1$)}
    \UnaryInfC{$\phi \lor \psi$}
\end{prooftree}
\begin{prooftree}
    \AxiomC{$\psi$}
    \RightLabel{($\lor i_2$)}
    \UnaryInfC{$\phi \lor \psi$}
\end{prooftree}

Elimination af 'eller' er den del sværere,
hvis vi har lyst til at konkludere $\alpha$ udfra $\phi \lor \psi$,
ved vi ikke om det er $\phi$, $\psi$ eller begge som er sande.
Vi bliver derfor nødt til at skabe 2 seperate beviser hvor vi antager $\phi$ og $\psi$ seperat og konkluderer $\alpha$ i begge tilfælde:

\begin{prooftree}
    \AxiomC{$\phi \lor \psi$}
    \AxiomC{$$\boxed{
        \begin{matrix}
            \phi \\
            \vdots \\
            \alpha
        \end{matrix}}$$}
    \AxiomC{$$\boxed{
        \begin{matrix}
            \psi \\
            \vdots \\
            \alpha
        \end{matrix}}$$}
    \RightLabel{($\lor e$)}
    \TrinaryInfC{$\alpha$}
\end{prooftree}

Vi vil også indføre et symbol som altid er falsk. Til dette bruger vi $\bot$. $\bot$ har inden introduktions regel, men kun en eliminations regel:
\begin{prooftree}
    \AxiomC{$\bot$}
    \RightLabel{($\bot e$)}
    \UnaryInfC{$\phi$}
\end{prooftree}
Denne regel giver at hvis du kan konkludere at noget er falsk, kan du udlede at enhver sætning er sand.

Reglerne for negation:
\begin{prooftree}
    \AxiomC{$$\boxed{
        \begin{matrix}
            \phi \\
            \vdots \\
            \bot
        \end{matrix}}$$}
    \RightLabel{($\lnot i$)}
    \UnaryInfC{$\lnot \phi$}
\end{prooftree}
Hvis du antager noget og kommer frem til at du kan konkludere falsk,
må det du antog være falsk. Det omvendte er derfor sandt.

\begin{prooftree}
    \AxiomC{$\lnot \lnot \phi$}
    \RightLabel{($\lnot \lnot e$)}
    \UnaryInfC{$\phi$}
\end{prooftree}
``minus minus giver plus''

\begin{prooftree}
    \AxiomC{$$\boxed{
        \begin{matrix}
            \phi \\
            \vdots \\
            \psi
        \end{matrix}}$$}
    \RightLabel{($\imp$ i)}
    \UnaryInfC{$\phi \imp \psi$}
\end{prooftree}
Hvis du kan bevise $\psi$ udfra $\phi$ så kan du konkludere $\phi \imp \psi$.

\begin{prooftree}
    \AxiomC{$\phi \imp \psi$}
    \AxiomC{$\phi$}
    \RightLabel{($\imp$ e)}
    \BinaryInfC{$\psi$}
\end{prooftree}
Hvis $\phi$ medfører $\psi$, og $\phi$ er sand, må $\psi$ også være sand.

Ækvivalent med ($\bimp$) er bare en forkortelse for at der er en implikation begge veje mellem udtrykkene. Det kan vi udtrykke ved reglerne
\par\noindent \begin{tabular}{@{}l@{}l@{}}
	\begin{minipage}{0.5\textwidth}
		\begin{prooftree}
		    \AxiomC{$\phi \imp \psi$}
		    \AxiomC{$\psi \imp \phi$}
		    \RightLabel{($\bimp$ i)}
		    \BinaryInfC{$\phi \bimp \psi$}
		\end{prooftree}
	\end{minipage}
	&
	\begin{minipage}{0.5\textwidth}
		\begin{prooftree}
		    \AxiomC{$\phi \bimp \psi$}
		    \RightLabel{($\bimp\text{ e}_\land$)}
		    \UnaryInfC{$\left(\phi \imp \psi\right) \land \left(\psi \imp \phi\right)$}
		\end{prooftree}
	\end{minipage}
\end{tabular}

\subsection{Beviser}
Da vi nu har alle reglerne på plads, kan vi bruge dem til at vise at andre sætninger er sande.
Hvis vi tager sætningen $p \imp q \vdash p \land r \imp q \land r$, ser vi at det ikke er et af vores aksiomer,
Men vi kan udfra vores aksiomer konkludere at sætningen er sand, ved at udføre et bevis for den:
\begin{prooftree}
    \AxiomC{$p \imp q$}
    \AxiomC{$p \land r$}
    \RightLabel{($\land e_1$)}
    \UnaryInfC{$p$}
    \RightLabel{($\imp$ e)}
    \BinaryInfC{$q$}
    \AxiomC{$p \land r$}
    \RightLabel{($\land e_2$)}
    \UnaryInfC{$r$}
    \RightLabel{($\land i$)}
    \BinaryInfC{$q \land r$}
\end{prooftree}

Dette ser en smule kludret ud, derfor er det bedre at bruge en form for linært bevis som ser således ud:

\begin{proofbox}
    \lbl{eks1_imp_beg}
    \[
        \lbl{eks1}
        \: p \land r        \= \text{Antagelse for $\imp i$ } \\
        \lbl{eks1_p}
        \: p                \= \land e_1, 1 \\
        \lbl{eks1_r}
        \: r                \= \land e_2, 1 \\
        \lbl{eks1_p_imp_q}
        \: p \imp q         \= \text{Antagelse} \\
        \lbl{eks1_q}
        \: q                \= \imp e \ref{eks1_p_imp_q}, \ref{eks1_p} \\
        \lbl{eks1_imp_end}
        \: q \land r        \= \land i \ref{eks1_q}, \ref{eks1_r}
    \]
    \: p \land r \imp q \land r \= \imp i \, \ref{eks1_imp_beg}-\ref{eks1_imp_end}
\end{proofbox}

\subsection{Eksempler på beviser}
\subsubsection{$p \imp q, q \imp r \vdash p \imp r$}
\begin{proofbox}
    \lbl{eks2_imp_box}
    \[
        \lbl{eks2_p}
        p \: p      \= \text{premisse for $\imp$ i} \\
        \lbl{eks2_piq}
        \: p \imp q \= \text{antagelse} \\
        \lbl{eks2_q}
        \: q        \= \imp e \ref{eks2_p}, \ref{eks2_piq} \\
        \lbl{eks2_qir}
        \: q \imp r \= \text{antagelse} \\
        \label{_eks2_imp_box_end}
        \: r        \= \imp e \ref{eks2_q} \\
    \]
    \: p \imp r     \= \imp i \ref{eks2_imp_box}-\ref{_eks2_imp_box_end}
\end{proofbox}

za
\subsubsection{$p \vdash q \imp p$}
Beviset her er lidt sjovt da $p$ klart er sandt, men selvom vi intet viden har omkring $q$,
kan $q$ stadig blive inkluderet i et udsagn.
\begin{proofbox}
    \lbl{eks3_qip}
    \[
        \lbl{eks2_q}
        q \: q      \= \text{premisse for $\imp$ i} \\
        \label{_eks3_qip_end}
        \: p        \= \text{Antagelse} \\
    \]
    \: q \imp p     \= \imp i \ref{eks3_qip}-\ref{_eks3_qip_end}
\end{proofbox}


%\begin{tabular}[l c r]
%    \begin{prooftree}
%        \AxiomC{$\phi$}
%        \RightLable{($\lor i$)}
%        \UnaryInfC{$\phi \lor \psi$}
%    \end{prooftree} &
%\end{tabular}

\ifdefined\startUdsagnslogik\end{document}\fi
