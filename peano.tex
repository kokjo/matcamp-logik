Formålet med peanos aksiom system er at beskrive hvad vi tænker på som naturlige tal.

De ser således ud:

\begin{prooftree}
    \AxiomC{}
    \UnaryInfC{$nat(0)$}
\end{prooftree}
Hvilket betyder: symbolet ``$0$'' er et naturligt tal (er en taotologi) og dette $0$ er et naturligt tal.


Med det vi har indført indtil videre er det eneste vi ved med sikkerhed omkring de naturlige tal er at der findes mindst et(``0''),
samt at dette tal er lig med sig selv.
Vi indføre derfor hvad det vil sige at noget kommer efter noget andet,
ved at indføre efterfølger prædikatet $S$.

Igen vil vi gøre således som vi intuitivt tænker på de naturlige tal.

Følgende gælder for $S$:

\begin{prooftree}
    \AxiomC{$\forall n \, nat(n)$}
    \UnaryInfC{$nat(S(n))$}
\end{prooftree}
Alle naturlige tal har en eterfølger som også selv er et naturligt tal.

\begin{prooftree}
    \AxiomC{$\forall n \, nat(n)$}
    \UnaryInfC{$\neg (S(n) = 0)$}
\end{prooftree}
Der findes intet naturligt tal hvis efterfølger er $0$(alle naturlige tal er ikke efterfølgere til $0$).

\begin{prooftree}
    \AxiomC{$\forall x, y \, nat(x) \land nat(y)$}
    \UnaryInfC{$S(x) = S(y)$}
\end{prooftree}
$2$ naturlige tal som er ens(dvs. opfylder ``$=$''-relationen) har også efterfølger som er ens.
Dette giver en sammenhængen mellem ``$=$'' og $S$ som vi ikke kunne sige noget om før,
indtil nu ``$=$'' og $S$ været uafhænige af hinnanden.


Det vigtigste aksiom er nok det vigtigste: Induktions aksiomet, hvilket giver os mulig hed for at introducere et
$\forall$-symbol med visse begrensninger:
\begin{prooftree}
    \AxiomC{$\phi(0) \land \forall a(\phi(a) \imp \phi(S(a)))$}
    \UnaryInfC{$\forall a \phi(a)$}
\end{prooftree}
Såfremt at en sætning($\phi$) er sand for $0$ og at hvis sætningen er sand for et naturligt tal
medføre at sætningen også er sand for dettes tals efterfølger,
kan vi konkludere at sætningen er sand for alle naturlige tal.

