Formålet med peanos aksiom system er at beskrive hvad vi tænker på som naturlige tal.

De ser således ud:

\begin{prooftree}
    \AxiomC{}
    \UnaryInfC{$nat(0)$}
\end{prooftree}
Hvilket betyder: symbolet ``$0$'' er et naturligt tal (er en taotologi) og dette $0$ er et naturligt tal.


Med det vi har indført indtil videre er det eneste vi ved med sikkerhed omkring de naturlige tal er at der findes mindst et(``0''),
samt at dette tal er lig med sig selv.
Vi indføre derfor hvad det vil sige at noget kommer efter noget andet,
ved at indføre efterfølger prædikatet $S$.

Igen vil vi gøre således som vi intuitivt tænker på de naturlige tal.

Følgende gælder for $S$:

\begin{prooftree}
    \AxiomC{$\forall n \, nat(n)$}
    \UnaryInfC{$nat(S(n))$}
\end{prooftree}
Alle naturlige tal har en eterfølger som også selv er et naturligt tal.

\begin{prooftree}
    \AxiomC{$\forall n \, nat(n)$}
    \UnaryInfC{$\neg (S(n) = 0)$}
\end{prooftree}
Der findes intet naturligt tal hvis efterfølger er $0$(alle naturlige tal er ikke efterfølgere til $0$).

\begin{prooftree}
    \AxiomC{$S(x) = S(y)$}
    \UnaryInfC{$x = y$}
\end{prooftree}
$2$ naturlige tal hvis efterfølgere er ens(dvs. opfylder ``$=$''-relationen) er også ens.
Dette giver en sammenhængen mellem ``$=$'' og $S$ som vi ikke kunne sige noget om før,
indtil nu ``$=$'' og $S$ været uafhænige af hinnanden. Dette aksiom giver også mulighedden for at fjerne $S$'er

Det vigtigste aksiom er nok det vigtigste: Induktions aksiomet, hvilket giver os mulighed for at introducere et
$\forall$-symbol med visse begrensninger:
\begin{prooftree}
    \AxiomC{$\phi(0)$}
    \AxiomC{$\forall a(\phi(a) \imp \phi(S(a)))$}
    \BinaryInfC{$\forall a \phi(a)$}
\end{prooftree}
Såfremt at en sætning($\phi$) er sand for $0$ og at hvis sætningen er sand for et naturligt tal
medføre at sætningen også er sand for dettes tals efterfølger,
kan vi konkludere at sætningen er sand for alle naturlige tal.

\subsection*{Aritmetik med naturlige tal}
Da vi nu har indført hvad det vil sige at være et naturligt tal,
kan vi udvide vores aksiom system med hvordan addition og multiplikation virker.

Aksiomerne for addition ser således ud:

\begin{prooftree}
    \AxiomC{$nat(a)$}
    \UnaryInfC{$a+0 = a$}
\end{prooftree}
Hvis man ligger 0 og et naturligt tal sammen vil dette 'nye' naturlige tal være det sammen som det første.

\begin{prooftree}
    \AxiomC{$nat(a)$}
    \AxiomC{$nat(b)$}
    \BinaryInfC{$S(a) + S(b) = S(S(a)) + b$}
\end{prooftree}


For multiplikation gælder der:
\begin{prooftree}
    \AxiomC{$nat(a)$}
    \UnaryInfC{$a \times S(0) = a$}
\end{prooftree}

\begin{prooftree}
    \AxiomC{$nat(a)$}
    \AxiomC{$nat(b)$}
    \BinaryInfC{$S(a) \times S(b) = S(a) \times b + S(a)$}
\end{prooftree}

\subsection*{bevis for at addition er kommutativ}

\begin{tabular}{l c r}
    1 & $nat(S(a))$ premisse \\
    2 & $S(a) + S(0)$ antaglese \\
    3 & $S(S(a) + 0)$ additions-egenskaben
    4 & $i$ \\
    2 & $nat(b)$ premisse \\

\end{tabular}
