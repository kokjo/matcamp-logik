Formålet med peanos aksiom system er at beskrive hvad vi tænker på som naturlige tal.

De ser således ud:

\[ 0 \in \nats \]
Hvilket betyder: ``$0$'' er indeholdt i mængden af naturlige tal, dvs. ``$0$'' er et naturligt tal.

For at vi kan tale om lighed bliver vi nød til at indføre det formelt først.
Det er derfor nødvendigt at definere hvordan ``$=$'' fungere, ved at beskrive hvilke ting det opfylder:

\[ \forall n \in \nats: n = n \]
``$=$''-relationen er refleksiv: Alle naturlige tal er lig med sig selv.

\[ \forall x,y \in \nats: x = y \bimp y = x \]
``$=$''-relationen er symmetrisk.

\[ \forall x,y,z \in \nats: x = y \land y = z \imp x = z \]
``$=$''-relationen er transitiv: Hvis x er ligmed y og y er ligmed z så må x være ligmed z.

\[ \forall y \forall x \in \nats: x = y \imp y \in \nats \]
``$=$''-relationen er begrænset til de naturlige tal. De naturlige tal er lukket under ``$=$''.
Det er vigtigt at bemærke at dette ikke udelukker at der findes ting som ikke er naturlige tal,
bare at vi ik kan bruge 'vores' ``$=$'' til andet end at sammenligne naturlige tal.


Med det vi har indført indtil videre er det eneste vi ved med sikkerhed omkring de naturlige tal er at der findes mindst et(``0''),
samt at dette tal er lig med sig selv. Vi indføre derfor hva det vil sige at noget kommer efter noget andet,
ved at indføre efter følger funktionen $S$. Igen vil vi gøre således som vi intuitivt tænker på de naturlige tal.

Følgende gælder for $S$:

\[ \forall n: n \in \nats \imp S(n) \in \nats \]
Hvis $n$ er et naturligt tal, så er det som kommer efter $n$ også et naturligt tal.

\[ \forall n: \neg (S(n) = 0) \]
Der findes intet naturligt tal hvis efterfølger er $0$.

\[ \forall x, y \in \nats: x = y \imp S(x) = S(y) \]
$2$ naturlige tal som er ens(dvs. opfylder ``$=$''-relationen) har også efterfølger som er ens.
Dette giver en sammenhængen mellem ``$=$'' og $S$ som vi ikke kunne sige noget om før,
indtil nu ``$=$'' og $S$ været uafhænige af hinnanden.

