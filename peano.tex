\ifx\preampleIncluded\undefined
\def\startPeano{}
\documentclass[12pt,a4paper,danish,twoside,reqno]{unf-compendium}

\usepackage[utf8]{inputenc}
\usepackage[danish]{babel}

\usepackage{amsmath, amsthm, amsfonts, amssymb}
\usepackage{bussproofs}
\usepackage{boxproof}
\usepackage{environ}
\usepackage{wasysym}
\usepackage{cleveref}
%\newtheorem{thm}{S{\ae}tning}
%\newtheorem{lem}{Lemma}
%\newtheorem{cor}{Korollar}

\theoremstyle{definition}
\newtheorem{dfn}{Definition}

\theoremstyle{remark}
\newtheorem{eks}{Eksempel}
\renewcommand{\phi}{\varphi}

\newcommand{\nats}{\ensuremath{\mathbb{N}}}
\newcommand{\reals}{\ensuremath{\mathbb{R}}}
\newcommand{\integers}{\ensuremath{\mathbb{Z}}}

\newcommand{\dx}{\,\mathrm{d} x}
\newcommand{\dz}{\,\mathrm{d} z}

\newcommand{\Uline}[1]{\underline{\underline{#1}}}
\newcommand{\vdashv}{\dashv\vdash}
\renewcommand{\imp}{\Rightarrow}
\newcommand{\bimp}{\Leftrightarrow}
\newcommand{\andL}{\wedge}
\newcommand{\orL}{\vee}
%\newcommand{\abs}[1]{\left|#1\right|}

\DeclareMathOperator{\Pfar}{far}
\DeclareMathOperator{\Pmor}{mor}
\DeclareMathOperator{\Pforaelder}{forælder}
\DeclareMathOperator{\Psoeskende}{søskende}
\DeclareMathOperator{\Phs}{hs}
\DeclareMathOperator{\Pfarmor}{farmor}
\DeclareMathOperator{\Pbedstemor}{bedstemor}
\DeclareMathOperator{\Pbedsteforaelder}{bedsteforælder}

\DeclareMathOperator{\Pvfu}{vfu}
\DeclareMathOperator{\pct}{\%}
\DeclareMathOperator{\BevisPar}{BevisPar}
\DeclareMathOperator{\Sub}{Sub}
\DeclareMathOperator{\Quine}{Quine}

\newif\ifsolution
\NewEnviron{solution}{\ifsolution{\par\noindent\textbf{Løsning.} }\expandafter\BODY\hfill$\square$\par\fi}
\theoremstyle{definition}\newtheorem{exercise}{Opgave}

\def\preampleIncluded{}

\begin{document}
\fi

\section{Peanos aksiomer}
Formålet med peanos aksiom system er at beskrive hvad vi tænker på som naturlige tal.
Den mest intuitive måske som vi tænker på naturlige tal er den som vi lærte ved at
tælle på fingre og vide hvilke tal som kommer efter andre tal.
f. eks er velkendt at $2$ kommer efter $1$, og der ikke kommer nogle (positive) tal før $0$.

Den måde hvor med vi lærte at ligge tal sammen var at man holdt $3$ fingre op på
den ene hånd og $2$ fingre på den anden,
og derefter hver gang man tog en finger ned på den ene hånd,
slog man en fingre op på den anden indtil at alle fingrene var taget ned på den ene hånd.
På den helt sammen og meget simple måde kommer vi også til at formulere de naturlige tal i logik.

Aksiomer vi har tænkt os at bruge ser således ud:

\begin{prooftree}
    \AxiomC{}
    \RightLabel{($p_{0b}$)}
    \UnaryInfC{$\forall n \neg (S(n) = 0)$}
\end{prooftree}
Der findes intet naturligt tal hvis efterfølger er $0$(alle naturlige tal er ikke efterfølgere til $0$).

\begin{prooftree}
    \AxiomC{$S(x) = S(y)$}
    \RightLabel{($p_{=} $)}
    \UnaryInfC{$x = y$}
\end{prooftree}
$2$ naturlige tal hvis efterfølgere er ens(dvs. opfylder ``$=$''-relationen) er også ens.
Dette giver en sammenhængen mellem ``$=$'' og $S$ som vi ikke kunne sige noget om før,
indtil nu ``$=$'' og $S$ været uafhænige af hinnanden. Dette aksiom giver også mulighedden for at fjerne $S$'er

Det vigtigste aksiom er nok det følgende: Induktions aksiomet, hvilket giver os mulighed for at introducere et
$\forall$-symbol med visse begrensninger:
\begin{prooftree}
    \AxiomC{$\phi(0)$}
    \AxiomC{$\forall a \, \phi(a) \imp \phi(S(a))$}
    \RightLabel{($p_I$)}
    \BinaryInfC{$\forall a \phi(a)$}
\end{prooftree}
Såfremt at en sætning($\phi$) er sand for $0$ og at hvis sætningen er sand for et naturligt tal
medføre at sætningen også er sand for dettes tals efterfølger,
Kan vi konkludere at sætningen er sand for alle naturlige tal.

Til sidst vil vi også tildele det velkendte mening til visse symboler for nemhedens skyld.
\begin{align*}
    \vdash 1 &= S(0) \\
    \vdash 2 &= S(S(0)) \\
    &\cdots\\
\end{align*}

\subsection*{Aritmetik med naturlige tal}
Da vi nu har indført hvad det vil sige at være et naturligt tal,
kan vi udvide vores aksiom system med hvordan addition og multiplikation virker.

Aksiomerne for addition ser således ud:

\begin{prooftree}
    \AxiomC{}
    \RightLabel{(${p+}_1$)}
    \UnaryInfC{$\forall a \, a+0 = a$}
\end{prooftree}
Hvis man ligger 0 og et naturligt tal sammen vil dette 'nye' naturlige tal være det sammen som det første.

\begin{prooftree}
    \AxiomC{}
    \RightLabel{(${p+}_2$)}
    \UnaryInfC{$\forall a \forall b \, a + S(b) = S(a+b)$}
\end{prooftree}
Additions egenskaben, beskriver hvordan vi tænker på addition intuitivt.

For multiplikation gælder der ligende regler:
\begin{prooftree}
    \AxiomC{}
    \RightLabel{(${p\times}_1$)}
    \UnaryInfC{$\forall a \, a \times 0 = 0$}
\end{prooftree}

\begin{prooftree}
    \AxiomC{}
    \RightLabel{(${p\times}_2$)}
    \UnaryInfC{$\forall a \forall b \, a \times S(b) = a \times b + a$}
\end{prooftree}


\subsection*{Bevis for at ``$0$'' er et neutral element}
For at $0$ skal være et additivt neutralelement skal det gælde at $\forall a \, (a+0 = a \land 0+a = a)$.
Vi ved allerede at ``$0$''neutral element hvis man bruger det fra højre side, det nemlig hvad aksiomet ${p+}_1$.
vi mangler derfor at vise at det er et venstre nuetral element.
For at gøre dette viser vi at predikatet $P(a) := 0+a = a$ er sand for alle naturlige tal.

\begin{proofbox}
    \: 0+0 = 0              \= \text{${p+}_1$ hvor $a$ er $0$} \\
    \lbl{base}
    \: P(0)
    \[
        \: P(a) \\
        \: 0+a = a          \= \text{antagelse for $\imp i$} \\
        \: S(0+a) = S(a)    \= \text{$=i$ og $=e$} \\
        \: 0+S(a) = S(a)    \= {p+}_1 \\
        \: P(S(a))
    \]
    \lbl{step}
    \: P(a) \imp P(S(a))    \= \imp i \\
    \: \forall a P(a)       \= p_I\, \ref{base}-\ref{step} 
    \: \forall a (0+a = a)
\end{proofbox}

Hvis vi ønsker at antage alle peanos aksiomer på engang bruger vi $PA \vdash \cdots$

\subsection*{Regler som kan udledes}
Med de aksiomer som vi nu har indført kan vi bevise en masse pæne egenskaber omkring de naturlige tal.
Vi vil dog ikke give bevis for disse sætninger da de alle er meget lange og en smule besværlige at vise.

Følgende sætninger må bruges uden bevis i jeres opgaver:
\begin{align*}
    &PA \vdash \forall a \forall b \, a+b = b+a &\text{Kommutivitet af $+$} \\
    &PA \vdash \forall a \forall b \, a \times b = b \times a &\text{Kommutivitet af $\times$} \\
    &PA \vdash \forall a \forall b \forall c \, a+(b+c) = (a+b)+c &\text{Associvitet af $+$} \\
    &PA \vdash \forall a \forall b \forall c \, a+(b+c) = (a+b)+c &\text{Associvitet af $\times$} \\
    &PA \vdash \forall a \forall b \forall c \, a\times (b +c) = a \times b + a \times c &\text{Den distributive lov}
\end{align*}

\ifdefined\startPeano\end{document}\fi
