\ifx\preampleIncluded\undefined
\def\startProjektTwo{}
\documentclass[12pt,a4paper,danish,twoside,reqno]{unf-compendium}

\usepackage[utf8]{inputenc}
\usepackage[danish]{babel}

\usepackage{amsmath, amsthm, amsfonts, amssymb}
\usepackage{bussproofs}
\usepackage{boxproof}
\usepackage{environ}
\usepackage{wasysym}
\newtheorem{thm}{Sætning}
\newtheorem{lem}{Lemma}
\newtheorem{cor}{Korollar}

\theoremstyle{definition}
\newtheorem{dfn}{Definition}

\theoremstyle{remark}
\newtheorem{eks}{Eksempel}
\renewcommand{\phi}{\varphi}

\newcommand{\nats}{\ensuremath{\mathbb{N}}}
\newcommand{\reals}{\ensuremath{\mathbb{R}}}
\newcommand{\integers}{\ensuremath{\mathbb{Z}}}

\newcommand{\dx}{\,\mathrm{d} x}
\newcommand{\dz}{\,\mathrm{d} z}

\newcommand{\Uline}[1]{\underline{\underline{#1}}}
\newcommand{\vdashv}{\dashv\vdash}
\renewcommand{\imp}{\Rightarrow}
\newcommand{\bimp}{\Leftrightarrow}
\newcommand{\andL}{\wedge}
\newcommand{\orL}{\vee}
%\newcommand{\abs}[1]{\left|#1\right|}

\DeclareMathOperator{\Pfar}{far}
\DeclareMathOperator{\Pmor}{mor}
\DeclareMathOperator{\Pforaelder}{forælder}
\DeclareMathOperator{\Psoeskende}{søskende}
\DeclareMathOperator{\Phs}{hs}
\DeclareMathOperator{\Pfarmor}{farmor}
\DeclareMathOperator{\Pbedstemor}{bedstemor}
\DeclareMathOperator{\Pbedsteforaelder}{bedsteforælder}

\newif\ifsolution
\NewEnviron{solution}{\ifsolution{\par\noindent\textbf{Løsning.} }\expandafter\BODY\hfill$\square$\par\fi}
\theoremstyle{definition}\newtheorem{exercise}{Opgave}

\def\preampleIncluded{}

\usepackage{enumitem}
\begin{document}
\solutiontrue
\fi

Vi vil i dette projekt betragte følgende udsagn i ren prædikatlogik:
\[
	\forall y \exists x P(x,y) \imp \exists x \forall y P(x,y)
\]

Hvis nu vi forestiller os en verden med kun et enkelt objekt i så må udsagnet være sandt, da
der så ikke er nogen forskel på at sige for-alle og der-eksisterer. Så vi vil derfor gå udfra
at sætningen udgør en ufuldstændighed (dette skal i ikke bevise).
Det vi gerne vil gøre her er at sætte så få krav som muligt til det logiske system som stadigvæk
gør det muligt at vise at sætningen er falsk.

\begin{enumerate}
	\item Start med at overvej hvad det egentligt er sætningen siger.
	\item \label{it2} Prøv uformelt at beskrive en situation hvor sætningen ikke holder.
\end{enumerate}
	Vi vil holde kravene vi stiller så generelle som muligt. F.eks. vil vi ikke kræve eksistensen af nogen navngivne objekter eller sætte nogen øvre grænser
	for hvor mange objekter der kan være. Specielt skal kravet vi stiller være kompatibelt med Peanos aksiomer. Prøv så at formulere et aksiom således at udsagnet bliver falsk.

	Nu vil vi gerne vise at vore tilføjede aksiom er tilstrækkeligt til at vi kan vise at udsagnet er falsk. Så vi skal prøve at formulere et prædikat $P$
	som virker som et modeksempel. Den sværeste del her er at vise at præmissen for udtrykket er opfyldt.

	I skal nu prøve at formalisere det i fandt frem til i \ref{it2}. 
\begin{enumerate}[resume]
	\item Forestil jeg at vores univers kun indeholder to elementer. Prøv at skriv et prædikat op der kan bruges som modeksempel.

	\item Nu skal der jo gerne kunne være mere end to elementer. Kan i omformulere jeres prædikat på en sådan måde at fortsat kan bruges som modeksempel?	
\end{enumerate}

Nu er vi så kommet til at vi skal prøve at bevise at udsagnet er falsk. Det er samlet ret langt så vi deler det op i mindre delsætninger.
Til at hjælpe jer må i bruge den her afledte sætning som vi kan kalde $\neq$ symmetri: 
\[\neg (t_0 = t_1) \vdash \neg(t_1 = t_0).\]
Det er selvfølgelig baseret på $=$ symmetrien der er vist i noterne
\[t_0=t_1 \vdash t_1=t_0.\]
Husk også specielt Loven om den Udeladte Midte:
\[
	\vdash \phi \lor \neg\phi
\]
I må selvføgelig også bruge alle de andre afledte sætninger i har set.
\begin{enumerate}
\item Start med at antage at vi kan bevise at venstresiden af udsagnet er sandt. Skriv så et formelt bevis op for at vi får en modstrid.
\item Beviset for at venstresiden er falsk kræver at man deler det op i forskellige tilfælde for $y$. Overvej hvordan man formelt kan lave et
bevis der er delt op i tilfælde. Antag så at hver af tilfældene er bevist, og skriv et formelt bevis op for at venstresiden er sand.
\item Lav nu et formelt bevis for hver af tilfældene.
\end{enumerate}

\ifdefined\startProjektTwo\end{document}\fi
